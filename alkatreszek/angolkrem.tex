\newpage
\section{Angolkrém} \label{sec:angolkem}

\subsubsection*{Hozzávalók kb. \qty{5}{\deci\l} krémhez}
\begin{itemize}
    \item \num{4} tojássárgája
    \item \num{5} evőkanál kristálycukor
    \item \qty{200}{\ml} tej
    \item \qty{200}{\ml} habtejszín
    \item \num{1} csipet só
    \item \num{1} rúd vanília
    \item \qty{20}{\g} puha \qty{82}{\percent}-os vaj
    \item \qty{20}{\g} finomliszt
\end{itemize}

Miután szétválasztottuk a tojásokat, a sárgájukat tegyük egy keverőtálba (lehetőleg egy olyanba, amely kibírja majd a gőz feletti hevítést), adjuk hozzá a kristálycukrot, majd habverővel verjük könnyű habbá.

Egy főzőedénybe öntsük a kimért tejet és a tejszínt. Vágjuk fel a vaníliarudat, kaparjuk ki a belsejét, majd a sóval együtt adjuk az edényhez. Kezdjük egy lassú tűzön melegíteni, miközben sűrűn kevergetjük, nehogy lekapjon.

Egy kis tálban eközben alaposan keverjük össze a kimért vajat és lisztet, hogy ezzel -- a fancy nevén beurre manié-nek nevezett dologgal -- majd sűríteni tudjuk a tejes keveréket. Ha már forrdogál a tejünk, adjuk hozzá a buerre manié-t, keverjük el, majd további \numrange{2}{3} percig főzzük.

Vékony sugárban csorgatva, folyamatos keverés mellett adjuk a kikevert tojássárgájához a tejes keverékünket. Egy lábosban forraljunk vizet, és helyezzük a lábos tetejére a keverőtálat, hogy a gőz felett hevíthessük (ugye mondtam én az első bekezdésben?!). Vigyázzunk, nehogy a keverőtál alja beleérjen a forró vízbe. Még mindig folyamatos keverés mellett melegítsük, amíg a krémünk \emph{be nem húz}, azaz el nem éri a kellő redukálást. Arra készüljünk, hogy a kész, lehűtött krém sűrűbb lesz, mint meleg állapotában. Ha van maghőmérőnk, akkor könnyebb megfejteni, hogy meddig kell ezt csinálnunk: kb. \qtyrange{85}{87}{\celsius}-t kell megcéloznunk.

Ezután már csak le kell hűtsük a krémünket. Ha tudunk jeges fürdőt csinálni egy még nagyobb keverőtálban, akkor abba helyezhetjük az eddig hevítésre használt tálat. Ha erre nincs mód/kedv, akkor hibátlanul le tud hűlni simán a pulton is a krém. Ha elérte a szobahőmérsékletet, rakjuk hűtőbe, és pár napig fogyasztható.


A recept alapja a~\cite{szell_angolkrem}.
