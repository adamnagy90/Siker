\newpage
\section{Lelovics májkrém} \label{sec:majkrem}

\subsubsection*{Hozzávalók kb. \qty{5}{\deci\l} krémhez}
\begin{itemize}
    \item \qty{150}{\ml} tej
    \item \num{2} evőkanál kacsazsír
    \item \num{1} teáskanál pirospaprika
    \item \num{1} teáskanál só
    \item \num{1} teáskanál bors
    \item \num{1} gerezd fokhagyma
    \item \num{0.25} gerezd vöröshagyma
\end{itemize}

Tisztítsuk meg a fehér izéktől, és kockázzuk fel a csirkemájat. A hagymákat aprítsuk fel. Minden hozzávalót tegyünk egy kis fazékba, és lassú tűzön kezdjük főzni.

Gyakori kevergetés és lassú bugyborgás mellett főzzük 1 órán keresztül.

A főzés végeztével botmixerrel vagy turmixszal homogenizáljuk a krémet. Ezután érdemes még egyszer felforralnunk a tűzön, hogy ne romoljon meg túl hamar.
