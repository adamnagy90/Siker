\newpage
\section{Fahéjas csiga} \label{sec:fahejas-csiga}

\subsubsection*{Hozzávalók kb. 18 kis fahéjas csigákhoz}
\begin{itemize}
    \item \qty{235}{\ml} langyos tej
    \item \qty{50}{\g} kristálycukor
    \item \qty{60}{\g} olvasztott és \qty{170}{\g} puha \qty{82}{\percent}-os vaj
    \item \qty{8}{\g} friss élesztő
    \item \qty{240}{\g} + \qty{60}{\g} Nagyi titka Kelt tészta süteményliszt
    \item \qty{160}{\g} barnacukor
    \item \qty{16}{\g} őrölt fahéj
    \item \qty{2}{\g} sütőpor
    \item \qty{5.5}{\g} só
\end{itemize}

A tejet, kristálycukrot, a vaj olvasztott részét és az élesztőt tegyük a dagasztóüstbe, keverjük el, majd hagyjuk pihenni \num{10} percig, hogy az élesztő felfuthasson.

Adjuk hozzá a liszt nagyobbik kimért részét, jól keverjük el, majd letakarva kelesszük \num{1} órán át. Amíg ez történik, vajazzuk ki a tűzálló tálakat, amikben majd kisütjük a csigákat.

Ekkor készítsük el a tölteléket is a csigákhoz: egy tálban keverjük össze alaposan a barnacukrot, a puha vajat és az őrölt fahéjt.

Amint letelt a kelesztési idő, adjuk a maradék lisztet az üsthöz a sütőporral és a sóval együtt, és dagasszuk készre a tésztát. Ha túl ragadós lenne, egy kevés lisztet adhatunk még hozzá, de próbáljuk minimalizálni ennek a mennyiségét.

Az így kapott tésztát belisztezett munkalapon először ujjainkkal nyújtsuk ki kicsit, majd nyújtófával formáljunk belőle kb.~\qty{5}{\mm} vastag téglalapot. Az előre összekevert tölteléket habkártyával vagy spatulával oszlassuk el a tészta tetején, majd a téglalap hosszabbik oldala körül tekerjük fel az egészet.

Az így kapott hengert fogselyemmel daraboljuk kb.~\qty{2}{\cm} vastag csigákra: csináljunk egy hurkot a henger körül, majd húzzuk meg a két végét a fogselyemnek, így kevésbé fog megnyomorodni a csigák kör alakja egy sima késes felvágáshoz képest. A csigákat tegyük a már kivajazott sütőtálakba figyelve arra, hogy még kelni fognak, így habár összeérhetnek, valamennyi helyet hagyjunk azért.

Ha később szeretnénk megsütni a csigákat, ezen a ponton lefóliázva lefagyaszthatóak -- tűzálló tál helyett ekkor rakhatjuk őket kivajazott aluminium tálcákra. A fagyasztott verzió még mindig finomabb lesz, mint bármi, amit eddig boltban tudtam venni, de azért lesz különbség a frisshez képest, és persze nem a jó irányba.

Ha frissen készítjük a csigákat, akkor takarjuk le őket, és további \numrange{35}{45} percig kelesszük őket. Ez idő alatt akár előkészíthetjük az öntetet is.

A készresütéshez melegítsük elő a sütőt \qty{180}{\celsius}-ra, és kb.~\num{25} percig hőkezeljük a csigákat. Próbáljunk meg várni \num{10} percet, mielőtt nekiállunk enni.

\subsubsection*{Hozzávalók az öntethez}
\begin{itemize}
    \item \qty{56}{\g} \qty{82}{\percent}-os olvasztott vaj
    \item \qty{250}{\g} krémsajt (mascarpone)
    \item \qty{120}{\ml} tej
    \item \qty{10}{\ml} vaníliakivonat opcionálisan
    \item \qty{200}{\g} porcukor
\end{itemize}

Először is: nem, nem lesz sajt íze. Tegyük egy tálba a porcukron kívül a többi összetevőt, majd egy habverővel keverjük el. Ezután fokozatosan adjuk hozzá a porcukrot folyamatos keverés mellett.

Ez a recept a~\cite{tasty_fahejas_csiga} átirata.
