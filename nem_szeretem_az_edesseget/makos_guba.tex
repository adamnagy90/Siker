\newpage
\section{Mákos guba} \label{sec:makos_guba}

\subsubsection*{Hozzávalók}
\begin{itemize}
    \item \qty{300}{\g} kalács (lásd \ref{sec:kalacs})
    \item \qty{5}{\deci\l} angolkrém (lásd \ref{sec:angolkem})
    \item \qty{50}{\g} \qty{82}{\percent}-os vaj
    \item \qty{100}{\g} darált mák
    \item \qty{100}{\g} porcukor
    \item \qty{2}{\deci\l} tej
\end{itemize}

kalács felszeletelése 2 centi vastagra, pirítás vajon, kockázás

100 g darált mák + 100 g porcukor kávédarálóba, darálás

angolkrémből kb 2 deci, ugyanennyi tejjel higítás

keverőtálba kalács, rá a lötty, kb 150 g a cukrosmákból, alapos keverés

sütő melegítés 185 fokra

hőálló tálba át, tetejére a maradék cukros mák, alufólia, sütőbe be 35 perc után fólia le, még 5 perc

maradék angolkrémmel leöntve tálaljuk
