\newpage
\section{Mákos guba} \label{sec:makos_guba}

\subsubsection*{Hozzávalók}
\begin{itemize}
    \item \qty{300}{\g} kalács (lásd: \fullref{sec:kalacs}, \pageref{sec:kalacs}. oldal)
    \item \qty{5}{\deci\l} angolkrém (lásd: \fullref{sec:angolkem}, \pageref{sec:angolkem}. oldal)
    \item \qty{50}{\g} \qty{82}{\percent}-os vaj
    \item \qty{2}{\deci\l} tej
    \item \qty{100}{\g} darált mák
    \item \qty{100}{\g} porcukor
\end{itemize}

Első lépésként szeleteljük fel a kalácsot kb. \qty{2}{\cm} vastag szeletekre. Hevítsünk egy serpenyőt, majd kevés vajon pirítsuk le a szeleteket, amiket aztán kockázzunk fel. Tegyük őket egy nagyobb keverőtálba, amiben kényelmesen elférnek.

Egy külön edénybe öntsünk kb.~\qty{2}{\deci\l}-t az angolkrémből, adjuk hozzá a kimért tejet, és alaposan keverjük el. Borítsuk nyakon ezzel a keverőtálban lévő kalácsot, és újra keverjük el. Ezen a ponton elkezdhetjük előmelegíteni a sütőt \qty{185}{\celsius}-ra.

Kávédaráló, kutter, aprító, vagy valami ilyesmi edényébe öntsünk mákot és a cukrot, és alaposan daráljuk le. Ennek nagy részével (kb. háromnegyede) szórjuk meg az immár angolkrémes kalácsot, és ismét keverjük el, hogy mindenhova egyenletesen jusson.

Vegyünk elő egy hőálló sütőtálat, és pakoljunk át bele mindent a keverőtálból, lehetőleg egy csepp krémet se pazaroljunk. A cukros mák maradékával szórjuk meg az így kapott entitást, takarjuk le alufóliával, majd tegyük a sütőbe. Kb. \num{35}~perc után vegyük le a fóliát, majd hagyjuk további \num{5} percet sülni.

Tálaláskor hideg angolkrémmel leöntve szolgáljuk fel, vagy hagyjuk, hogy mindenki eldöntse magának, miből mennyit kér. Ez mondjuk bármilyen ételre célszerű lehet.
