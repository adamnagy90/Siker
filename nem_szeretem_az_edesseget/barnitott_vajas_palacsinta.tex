\newpage
\section*{Barnított vajas palacsinta} \label{sec:barnitott-vajas-palacsinta}

\subsubsection*{Hozzávalók 10 db palacsintához}
\begin{itemize}
    \item \qty{150}{\g} \qty{82}{\percent}-os vaj
    \item \num{4} egész tojás
    \item \qty{20}{\g} kristálycukor
    \item \num{1} csipet só
    \item \qty{150}{\g} finomliszt
    \item \qty{200}{\ml} tej
    \item \qty{200}{\ml} ásványvíz/szóda
\end{itemize}

Kockázzuk fel a vajat, majd egy serpenyőben olvasszuk fel. Hevítsük nagy lángon, amíg fel nem habzik és szép barnás színe nem lesz. Ezután a serpenyőt húzzuk le a tűzről, és hagyjuk kicsit pihenni, amíg előkészítjük a tésztát.

Törjük be a tojásokat egy keverőtálba, adjuk hozzá a cukrot és a sót, majd egy habverővel keverjük el. Több részletben, átszitálva adjuk hozzá a lisztet is alapos keverés mellett. Meglepő módon minél több lisztet adunk hozzá, annál könnyebb lesz rendesen elkeverni, de akkor se öntsük bele egyből az egészet.

Keverjük hozzá a tejet és az ásványvizet is, majd végül vékony sugárban csorgatva adjuk hozzá a barnított vajat is, lehetőleg a leégett széndarabok nélkül. Ezután tegyük félre a tésztát pihenni kb.~\numrange{15}{20}~percre.

A sütéshez hevítsünk fel egy serpenyőt, olvasszunk bele nagyon kevés vajat, majd süssük ki a palacsintákat. A barnított vaj miatt a palacsinták között elvileg nem kell zsiradékot használnunk, a tésztában lévő elegendő lesz. A kisült palacsintákat pakoljuk egymásra, majd a végén fordítsuk meg az egész oszlopot fejjel lefelé. Most pedig Nutella elő!

Ez a recept a~\cite{szell_palacsinta} átirata.
