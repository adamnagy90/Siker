\newpage
\section{Túrógombóc} \label{sec:turogomboc}

\subsubsection*{Hozzávalók 20 db túrógombóchoz}
\begin{itemize}
    \item \num{3} tojás
    \item \num{3} + \num{6} + \num{2} evőkanál kristálycukor
    \item \num{3} csipet só
    \item \qty{1}{\kg} magas zsírtartalmú túró
    \item \num{3} evőkanál búzadara
    \item \qty{50}{\g} finomliszt
    \item \qty{150}{\g} panírmorzsa
    \item \qty{60}{\g} \qty{82}{\percent}-os vaj
    \item \qty{150}{\g} tejföl
    \item \num{1} evőkanál porcukor
\end{itemize}

Válasszuk szét a tojásokat két keverőtálba, majd a sárgájához adjuk hozzá az első kimért \num{3} evőkanál kristálycukrot, és keverjük el egy habverővel, miközben hozzáadunk egy csipet sót.

Törjük át a túrót egy fémszűrőn keresztül az eddigiekhez, és keverjük el. Adjuk hozzá a búzadarát is, majd kicsit tegyük félre.

Verjük laza habbá a tojásfehérjéket, és ezt is adjuk hozzá a másik keverőtálhoz, mindent jól elkeverve. A tojáshab össze fog törni, de ez ilyen. Lapítsuk bele a keveréket a tálba, majd közvetlen a felületén takarjuk le frissentartó fóliával. Ezután pihentetnünk kell hűtőben legalább \num{4}, de inkább \num{12} órán át.

A pihenés után, a hűtőből kivéve kicsit hagyjuk felmelegedni a túrógombócalapunkat. Forraljunk vizet, és adjuk hozzá a két csipet sót, illetve a \num{6} evőkanál kristálycukrot. Figyeljünk a víz hőmérsékletére, körülbelül \qty{90}{\celsius}-t kellene elérjünk.

A felmelegedett alapból formázzunk gombócokat, praktikusan egy sütőpapírra. A finomlisztet terítsük egy tálba, majd a gombócainkat forgassuk bele, hogy egy vékony réteg képződjön rajtuk. A gombócokat ezután tegyük a vízbe, majd fedő alatt minimális tűzön hagyjuk őket kb.~\qty{25}{percig}.

Egy serpenyőben pirítsuk le szárazon a morzsát, és mikor már aranybarnára váltott, csak akkor adjuk hozzá a vajat és a kristálycukrot. Ezután már csak bele kell ebbe forgatnunk a gombócokat.

Szokás szerint édestejföllel tálaljuk, ehhez előbb egy pohárban keverjük össze a tejfölt a porcukorral.

A recept a~\cite{fomenu_turogomboc} átirata.
