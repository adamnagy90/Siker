\newpage
\section*{Túrógombóc} \label{sec:turogomboc}

\subsubsection*{Hozzávalók 20 db túrógombóchoz}
\begin{itemize}
    \item \num{3} tojás
    \item \num{2} + \num{8} + \num{2} evőkanál kristálycukor
    \item \num{0.5} evőkanál só
    \item \qty{1}{\kg} magas zsírtartalmú túró
    \item \num{3} evőkanál búzadara
    \item \qty{50}{\g} finomliszt
    \item \qty{150}{\g} panírmorzsa
    \item \qty{60}{\g} \qty{82}{\percent}-os vaj
    \item \qty{150}{\g} tejföl
    \item \qty{130}{\g} porcukor
\end{itemize}

Válasszuk szét a tojásokat két keverőtálba, majd a sárgájához adjuk hozzá az első kimért \num{2} evőkanál kristálycukrot, és keverjük el egy habverővel, miközben hozzáadjuk a sót.

Törjük át a túrót egy fémszűrőn keresztül a keverékünkhöz, és keverjük el. Adjuk hozzá a búzadarát is, majd kicsit tegyük félre.

Verjük laza habbá a tojásfehérjéket, és ezt is adjuk hozzá a másik keverőtálhoz, mindent jól elkeverve. A tojáshab össze fog törni, de ez ilyen. Lapítsuk bele a keveréket a tálba, majd közvetlen a felületén takarjuk le frissentartó fóliával. Ezután pihentetnünk kell hűtőben legalább \num{4}, de inkább \num{24} órán át.

A hűtőből kivéve kicsit hagyjuk felmelegedni a túrógombócalapunkat. Forraljunk vizet, és adjuk hozzá a \num{8} evőkanál kristálycukrot.
