\newpage
\section*{Kovászos cipó} \label{sec:kovaszos-cipo}

\subsubsection*{Hozzávalók 1 db \qty{400}{\g}-os cipóhoz}
\begin{itemize}
    \item \qty{150}{\ml} langyos víz
    \item \num{1} evőkanál olívaolaj
    \item \qty{100}{\g} Caputo Manitoba liszt
    \item \qty{150}{\g} Caputo Nuvola liszt
    \item \num{0.5} zacskó Paneangeli Pasta madre disidratata con lievito szárított búzakovász
    \item \qty{5}{\g} cukor
    \item \num{1} csapott teáskanál só
\end{itemize}


A nedves hozzávalókat tegyük a dagasztóüstbe, majd miután a liszteket és a szárított kovászt összekevertük, adjuk azt is hozzá. Dagasszuk össze a tésztát, majd tegyük egy olajjal kikent tálba, és takarjuk le. \num{20} percenként nyújtsuk meg és hajtogassuk át a tésztát kézzel egymás után háromszor.

Tegyük a tésztát alaposan kilisztezett szakajtóba, takarjuk le nedves konyharuhával, és kelesszük egy órát a konyhapulton, vagy egy éjszakán át a hűtőben - ez esetben különösen ügyeljünk arra, hogy a tészta ne száradjon ki.

A pihentetés után melegítsük elő a sütőt egy pizzakővel együtt \qty{230}{\celsius}-ra. Tegyünk egy sütésálló serpenyőben vizet a sütő aljába, hogy gőz képződhessen. Borítsuk a cipót a szakajtóból a pizzakőre, majd egy éles késsel vágjuk be a tetejét ferdén. Süssük a kenyeret \num{20} percig, majd csendesítsük a sütőt \qty{210}{\celsius}-ra, és süssük készre a cipót.
