\newpage
\section*{Gyúrt tészta}

\subsection*{Alaptészta}
\subsubsection*{Hozzávalók kb. 6 adag tésztához}
\begin{itemize}
    \item \num{1} egész tojás
    \item \num{4} tojássárgája
    \item \num{1} evőkanál olivaolaj
    \item \num{1} teáskanál ecet
    \item \qty{250}{\g} Caputo Pasta fresa e gnocchi liszt
    \item \num{1} lapos kávéskanál só
\end{itemize}

\num{8} tojásos száraztészta? LOL. Ezzel a recepttel egy \num{20} tojásos friss gyúrt tésztát kaphatunk (ezt ugye liszt-kilogrammra fajlagosítva mérjük), ami tökéletes pl. lasagne vagy tagliatelle készítésére. A készételnél meg fogunk lepődni, hogy a tészta talán többet számít, mint a szósz, amit ráborítunk, és elvileg az ízt adja --- ezért tud működni egy aglio e olio is.

Öntsük a hozzávalókat a dagasztóüstbe, majd dagasszuk teljesen össze. Az elején a tészta túlságosan száraznak tűnhet, de legyünk türelemmel. Ha a lisztünk mégiscsak túl száraz lenne, pár csepp vizet adjunk hozzá. Csomagoljuk a tésztát folpackba, majd tegyük a hűtőbe legalább \num{4} órára, de inkább egy éjszakára. Ennek a végén gyurmaszerű, szép sárga tésztát kell kapjunk.~\cite{szell_gyurt_teszta}

\subsection*{Tészta készítése}
A hűtőből kivett tésztát vágjuk kezelhető darabokra, és mindig csak annyit tartsunk fólián kívül, amennyivel éppen dolgozunk, hogy ne száradjon ki. Szerencsésebb esetben tésztagép, ellenkezőleg nyújtófa segítségével nyújtsuk a tésztát kb. \qty{0.9}{\mm} vastagsúra, és szórjuk meg a felületét liszttel. Lasagne esetén simán vágjuk fel, tagliatelle esetén előbb tekerjük fel, majd úgy vágjuk fel csíkokra.

Az így kapott tésztát kiszáríthatjuk, lefagyaszthatjuk, vagy akár azonnal meg is főzhetjük, \numrange{1}{2} perc alatt al dentére kell készülnie.
