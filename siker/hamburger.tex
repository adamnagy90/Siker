\newpage
\section{Hamburger buci} \label{sec:hamburger-buci}

\subsubsection*{Hozzávalók 12 db kis méretű bucihoz}
\begin{itemize}
    \item \qty{200}{\ml} langyos víz
    \item \qty{25}{\g} friss sütőélesztő
    \item \qty{50}{\g} cukor
    \item \num{2} tojás
    \item \qty{28}{\g} puha \qty{82}{\percent}-os vaj
    \item \qty{420}{\g} Nagyi titka Kelt tészta süteményliszt
    \item \qty{8}{\g} só
    \item szezámmag
\end{itemize}

Az élesztőt morzsoljuk el a vízben, szórjunk bele kicsit a cukorból, keverjük el, majd szórjunk a tetejére a már kimért lisztből. Tegyük félre kb. \num{10} percre, hogy az élesztő felfuthasson.

A nedves hozzávalókkal kezdve tegyünk mindent a dagasztóüstbe, kivéve az egyik tojást és a szezámmagot, ezek csak a "díszítéshez" fognak kelleni (valójában a szezámmag az egyik fő indok, amiért az egészet csináljuk, nem?). Dagasszuk össze a tésztát, majd egy tálban letakarva kelesszük kb. \num{1} órán át.

Osszuk a tésztát \num{12} részre, hogy csini kis bucikat kapjunk, vagy \num{8}-ra, ha nagyobbakat szeretnénk. De szerintem a \num{12}-es osztás sokkal szebb lesz, no pressure. Gömbölyítsük fel a darabokat, majd helyezzük őket sütőpapírral bélelt sütőlemezre. Takarjuk le a bucikat, és hagyjuk őket kelni újabb kb. \num{1} órán át.

Melegítsük elő a sütőt \qty{190}{\celsius}-ra, és amennyiben nem gőzölős a sütőnk, helyezzünk az aljába egy hőálló tepsit vízzel töltve. Ez fog majd gondoskodni a megfelelően nedves légkörről, hogy a bucik ne repedjenek meg. Amíg melegedik a sütő, keverjük össze az eddig fel nem használt tojást egy kevés vízzel. Kenjük meg ezzel a bucik tetejét, majd szórjuk meg őket szezámmaggal.

Süssük a bucikat \numrange{12}{15} percig, vagy \numrange{15}{18} percig, ha csak \num{8} részre osztottuk a tésztánkat, míg szépen meg nem barnulnak. Biztos, ami biztos, mikor berakjuk a tepsit, virágspriccelővel is befújhatunk a sütőtérbe.

Ez a recept a~\cite{kab_hamburger} változtatás nélküli átirata --- a kommentektől eltekintve.
