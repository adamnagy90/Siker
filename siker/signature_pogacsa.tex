\newpage
\section*{Signature pogácsa} \label{sec:signature-pogacsa}

\subsection*{Alaptészta}
\subsubsection*{Hozzávalók kb. 3 tepsire}
\begin{itemize}
    \item \qty{150}{\ml} langyos tej
    \item \qty{25}{\g} friss sütőélesztő
    \item \qty{1}{csipet} cukor
    \item \qty{600}{\g} Nagyi titka Kelt tészta süteményliszt
    \item \qty{250}{\g} hideg \qty{82}{\percent}-os vaj
    \item \qty{20}{\g} só
    \item \qty{150}{g} \qty{20}{\percent}-os tejföl
    \item \num{1} tojás
    \item \qty{150}{\g} reszelt trappista
\end{itemize}

A sütőélesztőt morzsoljuk bele a tejbe, adjuk hozzá a cukrot, keverjük meg, szórjunk a tetejére kicsit a kimért lisztből, majd tegyük félre, hogy az élesztő felfuthasson.

A lisztbe kockázzuk bele a hideg vajat, majd teljesen morzsoljuk el. Ezzel amúgy a sikérháló kialakulását fékezzük, hogy omlós tésztát kaphassunk (ennek ellenére ne "Omlós tészta süteményliszt"-et használjunk, been there, done that, rosszabb lesz). Ezután adjuk hozzá a sót, és keverjük el. Szórjuk a dagasztógép üstjébe a tojást, a tejfölt, a felfutott élesztős tejet és a lisztben elmorzsolt vajat. Dagasszuk addig, míg a tészta összeáll, ekkor adjuk hozzá a reszelt trappista sajtot, majd dagasszuk készre a tésztát.

Ezen a ponton alaptésztánk elkészült, ha nagyobb adagot készítünk, ezt a pontot javasolnám a fagyasztásra. Folpackba csomagolva minőségromlás nélkül elrakható.

\subsection*{Készre sütés}
\subsubsection*{Hozzávalók kb. 3 tepsire}
\begin{itemize}
    \item \qty{40}{\g} olvasztott vaj
    \item \qty{50}{\g} reszelt parmezán
    \item \num{1} tojás
    \item \qty{1}{korty} tej
    \item \qty{250}{\g} reszelt trappista
    \item \qty{250}{\g} reszelt vörös cheddar
\end{itemize}

Amennyiben lefagyasztott alaptésztával indulunk, lassan hozzuk szobahőmérsékletre előbb a hűtőben, majd a konyhapulton.

A tésztát nyújtsuk kb.~\qty{6}{\mm} vastagságúra, kenjük meg az olvasztott vajjal és szórjuk meg reszelt parmezánnal. Hajtsuk fel a tészta kb. egyharmadát, és az így kapott új felületet szintén kenjük és szórjuk meg. Hajtsuk rá a maradék harmadot, ismét kenjük és szórjuk, majd hajtsuk félbe. Így elvileg hatrétegű tésztát kapunk, amit takarjunk le egy konyharuhával, majd hagyjuk pihenni \num{25} percig.

A pihentetés után a tésztát ismét nyújtsuk ki kb.~\qty{6}{\mm} vastagságúra, majd szaggassuk ki kis méretű pogácsaszaggatóval. Helyezzük a pogácsákat sütőpapírral bélelt sütőlemezre, majd a tejjel kikevert tojással kenjünk meg a tetejüket, és ismét hagyjuk őket pihenni \num{25} percig.

Miután megkeltek a pogácsák, kenjük meg tetejüket ismét a tojásos emulzióval, majd reszelt trappista és cheddar keverékével szórjuk meg bőven - kifejezetten ajánlott, hogy mellé is menjen a sütőpapírra! \qty{185}{\celsius}-ra előmelegített sütőben légkeverés mellett süssük \numrange{12}{14} percig, míg szépen meg nem pirulnak.

A recept inspirációja~\cite{szabi_pogi} volt.
