\newpage
\section*{Wannabe nápolyi pizza}

\subsection*{Alaptészta}
\subsubsection*{Hozzávalók 6 db pizzához}
\begin{itemize}
    \item \qty{325}{\ml} langyos víz
    \item \qty{10}{\g} friss sütőélesztő
    \item \qty{500}{\g} Caputo Nuvola pizzaliszt
    \item \qty{7}{\g} só
\end{itemize}

Morzsoljuk az élesztőt a vízbe, öntsük a dagasztóüstbe, és adjuk hozzá a lisztet. Fokozatosan adjuk hozzá a sót, majd dagasszuk legalább \num{5} percen át. Takarjuk le a tésztát, és hagyjuk pihenni kb. \num{30} percig.

Bontsuk a tésztát \num{6} egyenlő részre, majd gombócozzuk fel (hajtogassuk maga alá --- édesapám, saját maga felé dől! --- amíg a "felső" felület szépen ki nem feszül). Javasolt a pizzákat csak másnap készíteni, és lassú kelesztést használni, ehhez tegyük a gombócokat egy zárható üvegtálba, ellenkező esetben a pulton pihentessük őket fél órát.~\cite{gennaro_pizza}

\subsection*{Készre sütés}
\subsubsection*{Hozzávalók}
\begin{itemize}
    \item \num{1} egész San Marzano paradicsomkonzerv
    \item \num{2} evőkanál olivaolaj
    \item \num{1} gerezd fokhagyma
    \item kevés morzsolt bazsalikom
    \item durvaszemű semolina liszt
    \item mozzarella
    \item reszelt parmezán
\end{itemize}

Amennyiben a lassú kelesztést választottuk (helyes!), vegyük ki a tésztákat a hűtőből, és hagyjuk, hogy szobahőmérsékletűre melegedhessenek.

Amíg a tésztákra várunk, kezdjük el felmelegíteni a sütőt az elérhető maximális hőmérsékletre. Emiatt lesz csak wannabe a pizzánk, mert valószínűleg nem tudunk \qty{300}{\celsius} fölé menni sajnos. Ha mégis, hívjatok már meg! A sütőt állítsuk alsó-felső sütésre, és ha van pizzakövünk (legyen, sokkal jobb lesz az élet), azt melegítsük fel a sütővel együtt.

Készítsük elő a paradicsomszószt is: a paradicsomkonzervet és az olivaolajat öntsük egy mérőedénybe. Reszeljük bele a fokhagymát, illetve adjunk hozzá egy kevés morzsolt bazsalikomot. Botmixerrel keverjük/aprítsuk össze alaposan.

A pultot, ahol a tésztákat nyújtani szeretnénk, szórjuk meg semolinával. Egy adag tésztának mindkét felét lisztezzük meg, majd a tészta súlyát felhasználva kezdjük el nyújtani a kezünkben. Tegyük a pultra, ahol tovább nyújthatjuk, a liszten szépen kell tudnia csúsznia. Hagyjunk egy kicsit vastagabb szélet, de nyugodtan nyújtsuk a tésztát, a sikérhálónak köszönhetően elvileg csak nagyon nehezen szakadhat el.

Hintsük meg kevés olivaolajjal, majd \num{3} evőkanál paradicsomszószt oszlassunk el rajta. Őröljünk rá közepes szemcseméretű sót, majd tépkedjünk rá a lecsöpögtetett mozzarellából. Szórjuk meg reszelt parmezánnal, majd süssük kb. \num{5} percig (\qty{300}{\celsius} esetén). A sütőből kivéve még csepegtessünk rá kis olivaolajat.
