\chapter*{Sikér}

A könyv címadó fejezete a sikér nevet viseli, mely a glutén hivatalos megnevezése, gliadin és glutenin keveréke, és ez teszi ki a búza fehérjetartalmának kb.~\qty{80}{\percent}-át.~\cite{wiki_gluten} Gluténérzekenyek így sajnos lapozhatnak tovább, vagy megsüthetik a recepteket ajándékba.

Jellemzően minél magasabb a liszt sikértartalma, annál jobb minőségű termékről beszélhetünk. A sikérből alakul ki a tészta dagasztása, majd kelése során az úgynevezett sikérháló, ami a tészták rugalmasságát, alaktartóságát adja, ettől nem fog elszakadni a pizza nyújtáskor.

\subsection*{Belépő feltételek}
A fejezetben szereplő receptek feltételezik az alábbi konyhai eszközök elérhetőségét:
\begin{itemize}
    \item nyújtófa
    \item sütőpapír
    \item virágspriccelő
    \item legalább gramm pontosságú mérleg (de inkább tizedgrammos "kokainmérleg")
    \item dagasztógép
\end{itemize}

A dagasztógép természetesen helyettesíthető kézzel és izzadtsággal, a gramm pontosságú mérleg állítólag rutinnal, de én utóbbiban nem hiszek, előbbiben meg el lehet fáradni. A sütőpapír viszont nem helyettesíthető szilikonos sütőlappal, mert megváltoztatja a folyadékok mozgását.

\subsection*{Élesztő átváltás}
A receptekben az élesztő mennyisége friss élesztőként van megadva. Ezt macerás tárolni (hűtő, fuh) és viszonylag hamar meg is romlik, habár a legkönnyebb dolgozni vele. Friss élesztő helyett használhatunk aktív vagy instant szárított élesztőt is.

Aktív szárított élesztő esetén a megadott mennyiséget szorozzuk meg \num{0.4}-gyel. Ne felejtsük el újrahidratálni, és így a nedves hozzávalókhoz keverni.~\cite{kab_eleszto_atvaltas}

Instant szárírott élesztő esetén a hasonló faktor \num{0.33}, ezt a receptek száraz összetevőihez keverhetjük hidratálás nélkül, ne lepődjünk meg azonban, ha az "éledés" apró lokális foltokban fog megjelenni a tésztában.~\cite{kab_eleszto_atvaltas}

\newpage
\section{Signature pogácsa} \label{sec:signature-pogacsa}

\subsection*{Alaptészta}
\subsubsection*{Hozzávalók kb. 3 tepsire}
\begin{itemize}
    \item \qty{150}{\ml} langyos tej
    \item \qty{25}{\g} friss sütőélesztő
    \item \qty{1}{csipet} cukor
    \item \qty{600}{\g} Nagyi titka Kelt tészta süteményliszt
    \item \qty{250}{\g} hideg \qty{82}{\percent}-os vaj
    \item \qty{20}{\g} só
    \item \qty{150}{g} \qty{20}{\percent}-os tejföl
    \item \num{1} tojás
    \item \qty{150}{\g} reszelt trappista
\end{itemize}

A sütőélesztőt morzsoljuk bele a tejbe, adjuk hozzá a cukrot, keverjük meg, szórjunk a tetejére kicsit a kimért lisztből, majd tegyük félre, hogy az élesztő felfuthasson.

A lisztbe kockázzuk bele a hideg vajat, majd teljesen morzsoljuk el. Ezzel amúgy a sikérháló kialakulását fékezzük, hogy omlós tésztát kaphassunk (ennek ellenére ne "Omlós tészta süteményliszt"-et használjunk, been there, done that, rosszabb lesz). Ezután adjuk hozzá a sót, és keverjük el. Szórjuk a dagasztógép üstjébe a tojást, a tejfölt, a felfutott élesztős tejet és a lisztben elmorzsolt vajat. Dagasszuk addig, míg a tészta összeáll, ekkor adjuk hozzá a reszelt trappista sajtot, majd dagasszuk készre a tésztát.

Ezen a ponton alaptésztánk elkészült, ha nagyobb adagot készítünk, ezt a pontot javasolnám a fagyasztásra. Folpackba csomagolva minőségromlás nélkül elrakható.

\subsection*{Készre sütés}
\subsubsection*{Hozzávalók kb. 3 tepsire}
\begin{itemize}
    \item \qty{40}{\g} olvasztott vaj
    \item \qty{50}{\g} reszelt parmezán
    \item \num{1} tojás
    \item \qty{1}{korty} tej
    \item \qty{250}{\g} reszelt trappista
    \item \qty{250}{\g} reszelt vörös cheddar
\end{itemize}

Amennyiben lefagyasztott alaptésztával indulunk, lassan hozzuk szobahőmérsékletre előbb a hűtőben, majd a konyhapulton.

A tésztát nyújtsuk kb.~\qty{6}{\mm} vastagságúra, kenjük meg az olvasztott vajjal és szórjuk meg reszelt parmezánnal. Hajtsuk fel a tészta kb. egyharmadát, és az így kapott új felületet szintén kenjük és szórjuk meg. Hajtsuk rá a maradék harmadot, ismét kenjük és szórjuk, majd hajtsuk félbe. Így elvileg hatrétegű tésztát kapunk, amit takarjunk le egy konyharuhával, majd hagyjuk pihenni \num{25} percig.

A pihentetés után a tésztát ismét nyújtsuk ki kb.~\qty{6}{\mm} vastagságúra, majd szaggassuk ki kis méretű pogácsaszaggatóval. Helyezzük a pogácsákat sütőpapírral bélelt sütőlemezre, majd a tejjel kikevert tojással kenjünk meg a tetejüket, és ismét hagyjuk őket pihenni \num{25} percig.

Miután megkeltek a pogácsák, kenjük meg tetejüket ismét a tojásos emulzióval, majd reszelt trappista és cheddar keverékével szórjuk meg bőven - kifejezetten ajánlott, hogy mellé is menjen a sütőpapírra! \qty{185}{\celsius}-ra előmelegített sütőben légkeverés mellett süssük \numrange{12}{14} percig, míg szépen meg nem pirulnak.

A recept inspirációja~\cite{szabi_pogi} volt.

\newpage
\section*{Kifli/perec}

\subsection*{Alaptészta}
\subsubsection*{Hozzávalók}
\begin{itemize}
    \item \qty{67.5}{\ml} langyos víz
    \item \qty{75}{\ml} langyos tej
    \item \qty{10}{\g} friss sütőélesztő
    \item \qty{7.5}{csipet} cukor
    \item \qty{250}{\g} Nagyi titka Kelt tészta süteményliszt
    \item \qty{25}{\g} puha \qty{82}{\percent}-os vaj
    \item \qty{7.5}{\g} só
\end{itemize}


\newpage
\section*{Hamburger buci} \label{sec:hamburger-buci}

\subsubsection*{Hozzávalók 12 db kis méretű bucihoz}
\begin{itemize}
    \item \qty{200}{\ml} langyos víz
    \item \qty{25}{\g} friss sütőélesztő
    \item \qty{50}{\g} cukor
    \item \num{2} tojás
    \item \qty{28}{\g} puha \qty{82}{\percent}-os vaj
    \item \qty{420}{\g} Nagyi titka Kelt tészta süteményliszt
    \item \qty{8}{\g} só
    \item szezámmag
\end{itemize}

Az élesztőt morzsoljuk el a vízben, szórjunk bele kicsit a cukorból, keverjük el, majd szórjunk a tetejére a már kimért lisztből. Tegyük félre kb. \num{10} percre, hogy az élesztő felfuthasson.

A nedves hozzávalókkal kezdve tegyünk mindent a dagasztóüstbe, kivéve az egyik tojást és a szezámmagot, ezek csak a "díszítéshez" fognak kelleni (valójában a szezámmag az egyik fő indok, amiért az egészet csináljuk, nem?). Dagasszuk össze a tésztát, majd egy tálban letakarva kelesszük kb. \num{1} órán át.

Osszuk a tésztát \num{12} részre, hogy csini kis bucikat kapjunk, vagy \num{8}-ra, ha nagyobbakat szeretnénk. De szerintem a \num{12}-es osztás sokkal szebb lesz, no pressure. Gömbölyítsük fel a darabokat, majd helyezzük őket sütőpapírral bélelt sütőlemezre. Takarjuk le a bucikat, és hagyjuk őket kelni újabb kb. \num{1} órán át.

Melegítsük elő a sütőt \qty{190}{\celsius}-ra, és amennyiben nem gőzölős a sütőnk, helyezzünk az aljába egy hőálló tepsit vízzel töltve. Ez fog majd gondoskodni a megfelelően nedves légkörről, hogy a bucik ne repedjenek meg. Amíg melegedik a sütő, keverjük össze az eddig fel nem használt tojást egy kevés vízzel. Kenjük meg ezzel a bucik tetejét, majd szórjuk meg őket szezámmaggal.

Süssük a bucikat \numrange{12}{15} percig, vagy \numrange{15}{18} percig, ha csak \num{8} részre osztottuk a tésztánkat, míg szépen meg nem barnulnak. Biztos, ami biztos, mikor berakjuk a tepsit, virágspriccelővel is befújhatunk a sütőtérbe.

Ez a recept a~\cite{kab_hamburger} változtatás nélküli átirata --- a kommentektől eltekintve.

\newpage
\section*{Wannabe nápolyi pizza}

\subsection*{Alaptészta}
\subsubsection*{Hozzávalók 6 db pizzához}
\begin{itemize}
    \item \qty{325}{\ml} langyos víz
    \item \qty{10}{\g} friss sütőélesztő
    \item \qty{500}{\g} Caputo Nuvola pizzaliszt
    \item \qty{7}{\g} só
\end{itemize}

Morzsoljuk az élesztőt a vízbe, öntsük a dagasztóüstbe, és adjuk hozzá a lisztet. Fokozatosan adjuk hozzá a sót, majd dagasszuk legalább \num{5} percen át. Takarjuk le a tésztát, és hagyjuk pihenni kb. \num{30} percig.

Bontsuk a tésztát \num{6} egyenlő részre, majd gombócozzuk fel (hajtogassuk maga alá --- édesapám, saját maga felé dől! --- amíg a "felső" felület szépen ki nem feszül). Javasolt a pizzákat csak másnap készíteni, és lassú kelesztést használni, ehhez tegyük a gombócokat egy zárható üvegtálba, ellenkező esetben a pulton pihentessük őket fél órát.~\cite{gennaro_pizza}

\subsection*{Készre sütés}
\subsubsection*{Hozzávalók}
\begin{itemize}
    \item \num{1} egész San Marzano paradicsomkonzerv
    \item \num{2} evőkanál olivaolaj
    \item \num{1} gerezd fokhagyma
    \item kevés morzsolt bazsalikom
    \item durvaszemű semolina liszt
    \item mozzarella
    \item reszelt parmezán
\end{itemize}

Amennyiben a lassú kelesztést választottuk (helyes!), vegyük ki a tésztákat a hűtőből, és hagyjuk, hogy szobahőmérsékletűre melegedhessenek.

Amíg a tésztákra várunk, kezdjük el felmelegíteni a sütőt az elérhető maximális hőmérsékletre. Emiatt lesz csak wannabe a pizzánk, mert valószínűleg nem tudunk \qty{300}{\celsius} fölé menni sajnos. Ha mégis, hívjatok már meg! A sütőt állítsuk alsó-felső sütésre, és ha van pizzakövünk (legyen, sokkal jobb lesz az élet), azt melegítsük fel a sütővel együtt.

Készítsük elő a paradicsomszószt is: a paradicsomkonzervet és az olivaolajat öntsük egy mérőedénybe. Reszeljük bele a fokhagymát, illetve adjunk hozzá egy kevés morzsolt bazsalikomot. Botmixerrel keverjük/aprítsuk össze alaposan.

A pultot, ahol a tésztákat nyújtani szeretnénk, szórjuk meg semolinával. Egy adag tésztának mindkét felét lisztezzük meg, majd a tészta súlyát felhasználva kezdjük el nyújtani a kezünkben. Tegyük a pultra, ahol tovább nyújthatjuk, a liszten szépen kell tudnia csúsznia. Hagyjunk egy kicsit vastagabb szélet, de nyugodtan nyújtsuk a tésztát, a sikérhálónak köszönhetően elvileg csak nagyon nehezen szakadhat el.

Hintsük meg kevés olivaolajjal, majd \num{3} evőkanál paradicsomszószt oszlassunk el rajta. Őröljünk rá közepes szemcseméretű sót, majd tépkedjünk rá a lecsöpögtetett mozzarellából. Szórjuk meg reszelt parmezánnal, majd süssük kb. \num{5} percig (\qty{300}{\celsius} esetén). A sütőből kivéve még csepegtessünk rá kis olivaolajat.

\newpage
\section*{Gyúrt tészta}

\subsection*{Alaptészta}
\subsubsection*{Hozzávalók kb. 6 adag tésztához}
\begin{itemize}
    \item \num{1} egész tojás
    \item \num{4} tojássárgája
    \item \num{1} evőkanál olivaolaj
    \item \num{1} teáskanál ecet
    \item \qty{250}{\g} Caputo Pasta fresa e gnocchi liszt
    \item \num{1} lapos kávéskanál só
\end{itemize}

\num{8} tojásos száraztészta? LOL. Ezzel a recepttel egy \num{20} tojásos friss gyúrt tésztát kaphatunk (ezt ugye liszt-kilogrammra fajlagosítva mérjük), ami tökéletes pl. lasagne vagy tagliatelle készítésére. A készételnél meg fogunk lepődni, hogy a tészta talán többet számít, mint a szósz, amit ráborítunk, és elvileg az ízt adja --- ezért tud működni egy aglio e olio is.

Öntsünk a hozzávalókat a dagasztóüstbe, majd dagasszuk teljesen össze. Az elején a tészta túlságosan száraznak tűnhet, de legyünk türelemmel. Ha a lisztünk mégiscsak túl száraz lenne, pár csepp vizet adjunk hozzá. Csomagoljuk a tésztát folpackba, majd tegyük a hűtőbe legalább \num{4} órára, de inkább egy éjszakára. Ennek a végén gyurmaszerű, szép sárga tésztát kell kapjunk.~\cite{szell_gyurt_teszta}

\subsection*{Tészta készítése}
A hűtőből kivett tésztát vágjuk kezelhető darabokra, és mindig csak annyit tartsunk fólián kívül, amennyivel éppen dolgozunk, hogy ne száradjon ki. Szerencsésebb esetben tésztagép, ellenkezőleg nyújtófa segítségével nyújtsuk a tésztát kb. \qty{0.9}{\mm} vastagsúra, és szórjuk meg a felületét liszttel. Lasagne esetén simán vágjuk fel, tagliatelle esetén előbb tekerjük fel, majd úgy vágjuk fel csíkokra.

Az így kapott tésztát kiszáríthatjuk, lefagyaszthatjuk, vagy akár azonnal meg is főzhetjük, \numrange{1}{2} perc alatt al dentére kell készülnie.


% \section*{Kenyérke}
