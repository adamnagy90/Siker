\newpage
\section{Kifli/perec} \label{sec:kifli-perec}

\subsection*{Alaptészta}
\subsubsection*{Hozzávalók kb. 7 kiflire/perecre}
\begin{itemize}
    \item \qty{67.5}{\ml} langyos víz
    \item \qty{75}{\ml} langyos tej
    \item \qty{10}{\g} friss sütőélesztő
    \item \qty{7.5}{\g} cukor
    \item \qty{250}{\g} Nagyi titka Kelt tészta süteményliszt
    \item \qty{25}{\g} puha \qty{82}{\percent}-os vaj
    \item \qty{7.5}{\g} só
\end{itemize}

Morzsoljuk el a sütőélesztőt a víz és tej keverékében, adjuk hozzá a cukrot, majd keverjük el. Szórjunk a tetejére a kimért lisztből, és tegyük félre kb. \num{10} percre, hogy felfuthasson.

Öntsük az élesztős keveréket a dagasztóüstbe, adjuk hozzá a maradék lisztet és a vajat, és kezdjük el dagasztani. Amint a tészta összeállt, adjuk hozzá a sót, és dagasszuk addig, míg a tészta el nem válik az edény falától. Ezután helyezzük egy olivaolajjal vékonyan kikent tálba, és letakarva pihentessük kb. \num{25} percig.

A pihentetett tésztát osszuk fel \num{7} darab kb. \qty{65}{\g}-os részre, majd ezeket gömbölyítsük fel. Virágspriccelővel nedvesítsük meg a gombócok tetejét, majd takarjuk le őket egy szintén nedves konyharuhával. Hagyjuk őket pihenni kb. \num{25} percig, hogy a sikérszálak elernyedhessenek.~\cite{szabi_kifli}

\subsection*{Kiflik készítése}
Tenyérrel lapítsuk ki kicsit a lisztezetlen munkalapon a kiválasztott gombócot, majd nyújtófával nyújtsuk háromszög alakúra. A háromszög egyik élénél kezdve tekerjük fel, miközben folyamatosan feszítve tartjuk a tésztát. Helyezzük sütőpapírral bélelt mély sütőtepsibe, majd virágspriccelővel tartsuk nedvesen a felületét.

Amint minden kiflit megformáztunk, a tepsit fedjük be folpackkal, és hagyjuk őket kelni kb. \num{25} percig.

A pihentetés után vegyük le a fóliát, ekkor megszórhatjuk a kifliket nagyszemű sóval, köménymaggal (fúj) vagy akár reszelt sajttal is. \qty{220}{\celsius}-ra előmelegített sütőben süssük készre a kifliket nedves légtérben (virágspriccelő, gőzölős sütő luxiban) \numrange{12}{14} perc alatt. Az elkészült kifliket is lespriccelhejtük kicsit vízzel, ezzel fényesebbek lesznek.~\cite{szabi_kifli}

\subsection*{Perecek készítése}
Tenyérrel lapítsuk ki kicsit a lisztezetlen munkalapon a kiválasztott gombócot (hiba a Mátrixban?), hajtsuk félbe, majd sodorjuk kb. \qty{8}{\mm} vastagságúra. Tegyük ezt meg az összes gombóccal, amiből perecet szeretnénk sütni, hogy két lépésben nyújthassuk ki a tésztát. Második körben sodorjuk kb. \qty{5}{\mm} vastagságúra, a kis hurka lehet a közepén kicsit vastagabb, mint a szélein.

Hajtsuk félbe a hurkát, kb. félúton keresztezzük a szárakat, és tekerjünk rajtuk egyet. Az így keletkezett hurkot kicsit nyújtsuk ki, majd hajtsuk fel rá a szárakat, és tapasszuk őket két oldalra. Ezzel elvileg létrejött a fejjel-lefelé álló perecünk, amit már csak meg kell fordítanunk, habár így leírás alapján elismerem, hogy ezt nehéz lehet megcsinálni. A kiflihez hasonlóan helyezzük sütőpapírral bélelt mély sütőtepsibe, és tartsuk nedvesen a felületét.

Amint mindegyik elkészült, a tepsit itt is fedjük be folpackkal, és pihentessük \num{25} percig.

A pihentetés után tegyük \qty{190}{\celsius}-ra előmelegített sütőbe \num{14} percre. Ha pancsos perecet szeretnénk készíteni, keverjünk ekkor össze \num{2} evőkanál lisztet \num{1} evőkanál sóval és kb. ugyannennyi vízzel, hogy egy alacsony viszkozitású emulziót kapjunk. Miután letelt a \num{14} perc, vegyük ki a pereceket a sütőből, és csurgassuk a tetejükre a pancsot. Ezután \qty{200}{\celsius}-on süssük őket további \num{4} percig.~\cite{szabi_perec}
