\newpage
\section{Kalács} \label{sec:kalacs}

\subsubsection*{Hozzávalók 2 db \qty{300}{\g}-os kalácshoz}
\begin{itemize}
    \item \qty{187.5}{\ml} langyos tej
    \item \qty{20}{\g} friss sütőélesztő
    \item \num{1} tojássárgája
    \item \qty{375}{\g} Nagyi titka Kelt tészta süteményliszt
    \item \qty{60}{\g} cukor
    \item \num{0.5} narancs héja
    \item \qty{7.5}{\g} só
    \item \qty{37.5}{\g} puha \qty{82}{\percent}-os vaj
    \item \num{1} tojás
\end{itemize}

Adjuk a tejet, az élesztőt és a tojássárgáját a dagasztóüstbe, és kevertessük el. A szokáshoz képest itt most nem kell megvárjuk, míg az élesztő felfut, lesz még elég ideje a tésztánknak.

Borítsuk a lisztet és a sót az üstbe. A narancs héját apró lyukú reszelőn reszeljük bele a kristálycukorba, és keverjük el. Adjuk ezt is a keverőtálba a sóval együtt. Kezdjük dagasztani az egész mindenséget, majd eközben folyamatosan adagoljuk hozzá a vajat. Dagasszuk egészen addig, míg az üst falától el nem válik a tésztánk. Miután ez megtörtént, helyezzük egy tiszta tálba, és letakarva pihentessük, míg duplájára nem kel -- ez kb. \num{45}~percbe fog telni.

Lisztezetlen munkalapon bontsuk a tésztánkat 6 egyenlő részre (kb. \qty{120}{\g} lesz, ha minden igaz), és ezekből formáljunk gombócokat. Azért 6, mert így két darab háromszálas kalácsunk lesz a végén. Letakarva pihentessük őket további \num{15}~percig.

A kis gombócokat lapítsuk ki a tenyerünkkel, hajtsuk félbe, majd sodorjunk belőle kb. \qty{20}{\cm} hosszú hasas kígyókat -- azaz középen legyenek vastagabbak. Tegyük ezt meg mindegyikkel, hogy picit pihenhessenek, majd nyújtsuk őket tovább, még hosszabbra. A művelet végére a tésztaszálak legvastagabb része legyen kb.~\qty{2}{\cm} vastag.

És most nagy levegő, következik a fonás! Fektessünk egy szálat a pultra, majd rá keresztbe egy másikat, hogy egy X-et formázzanak. Tegyük erre rá a harmadik szálunkat az elsővel (ami ugye egyben az alsó is) párhuzamosan. Ezután a \emph{szokásos} módon, az első szállal kezdve fonjuk végig a kalács egyik felét. Ha elértünk a végére, a szálakat kicsit csípjük össze, nehogy kinyíljanak a további kelés vagy sütés folyamán. Ezután fordítsuk meg a kalácsot (praktikusan fejjel lefelé), hogy a három szabad tésztaszál nézzen felénk, és ezt a felét is fonjuk végig. Ha elsőre nem lett túl szép, semmi gond, ez a recept úgyis két kalácsról szól.

Helyezzük az immár gyönyörűen megfont kalácsokat sütőpapírral bélelt sütőlemezre, majd egy tálkában kikevert egész tojással hosszú mozdulatokkal kenjük meg őket. Pihentessük őket így kb.~\num{15}~percig. Amíg a kalácsok pihennek, melegítsük elő a sütőt alsó-felső sütési módban \qty{185}{\celsius}-ra. A korábbi receptek alapján szinte már szokásosnak mondhatóan, ha tudunk gőzt csinálni a sütőtérben, hajrá.

A pihenő lejárta után kenjük meg újra a kalácsok felületét a tojással, tegyük őket a sütőbe, majd csendesítsük a beállított hőfokot \qty{180}{\celsius}-ra. Körülbelül \numrange{22}{24}~perc alatt készre sülnek a kalácsok.

Ez a recept~\cite{szabi_kalacs} alapján készült.
