\newpage
\section{Kalács} \label{sec:kalacs}

\subsubsection*{Hozzávalók 2 db \qty{300}{\g}-os kalácshoz}
\begin{itemize}
    \item \qty{187.5}{\ml} langyos tej
    \item \qty{20}{\g} friss sütőélesztő
    \item \num{1} tojássárgája
    \item \qty{375}{\g} Nagyi titka Kelt tészta süteményliszt
    \item \qty{60}{\g} cukor
    \item \num{0.5} narancs héja
    \item \qty{7.5}{\g} só
    \item \qty{37.5}{\g} puha \qty{82}{\percent}-os vaj
    \item \num{1} tojás
\end{itemize}

Adjuk a tejet, az élesztőt és a tojássárgáját a dagasztóüstbe, és kevertessük el. A szokáshoz képest itt most nem kell megvárjuk, míg az élesztő felfut, lesz még elég ideje a tésztánknak.

Borítsuk a lisztet és a sót az üstbe. A narancs héját apró lyukú reszelőn reszeljük bele a kristálycukorba, és keverjük el. Adjuk ezt is a keverőtálba a sóval együtt. Kezdjük dagasztani az egész mindenséget, majd eközben folyamatosan adagoljuk hozzá a vajat. Dagasszuk egészen addig, míg az üst falától el nem válik a tésztánk. Miután ez megtörtént, helyezzük egy tiszta tálba, és letakarva pihentessük, míg duplájára nem kel -- ez kb. \num{45}~percbe fog telni.

Lisztezetlen munkalapon bontsuk a tésztánkat 6 egyenlő részre (kb. \qty{120}{\g} lesz, ha minden igaz), és ezekből formáljunk gombócokat. Azért 6, mert így két darab háromszálas kalácsunk lesz a végén. Letakarva pihentessük őket további \num{15}~percig.

lapítás, félbehajtás, nyújtás hasas szálra, majd hosszabbra

X két szálból, elsővel párhuzamos harmadik, fonás, felfordítás, fonás a másik végéig

sütőpapírra, felvert tojással hosszú mozdulatokkal, pihentetés (15 perc)

újrakenés, előmelegített sütő 185, majd 180 fokra, 22-24 perc



Ez a recept~\cite{szabi_kalacs} alapján készült.
